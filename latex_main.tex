\documentclass[12pt,a4paper]{article}
\usepackage{amsmath}
\usepackage{algorithm}
\usepackage{algpseudocode}
\usepackage{graphicx}
\usepackage{xcolor}    
\usepackage{soul} 
\usepackage{cite} 
\usepackage[T1]{fontenc}
\title{SSL Latex Assignment}
\author{SUNNY RAJ \\ \\CS23BT054 \\ \\ Department of Computer Science, IIT Dharwad}
\date{\today}

\begin{document}
	\maketitle
	\pagebreak
	\begin{itemize}
		\item[] \textbf{Contents}
		\begin{enumerate}
			\item[1] \textbf{Introduction} \hfill{3}
			\item[2] \textbf{Mathematics} \hfill{4}
			\begin{enumerate}
				\item[2.1] Writing Trigonometric equations \dotfill{4}
				\begin{enumerate}
					\item[2.1.1] Some trigonometric identities \dotfill{4}
				\end{enumerate}
				\item[2.2] Matrices \dotfill{4}
				\item[2.3] Radicals \dotfill{4}
				\item[2.4] Integration \dotfill{5}
				\item[2.5] Summation \dotfill{5}
				\item[2.6] Differentiation \dotfill{6}
				\item[2.7] Nested brackets \dotfill{6}
			\end{enumerate}
			\item[3] \textbf{Tables and Figures} \dotfill{7}
			\item[4] \textbf{Numberings} \dotfill{8}
			\begin{itemize}
				\item[]
				\begin{itemize}
					\item[]
			\begin{enumerate}
						\item[4.0.1] Unordered (Bullet) Lists \dotfill{8}
						\item[4.0.2] Ordered (Numbered) Lists \dotfill{8}
						\item[4.0.3] Using Roman Numerals \dotfill{8}
						\item[4.0.4] Using Letters \dotfill{8}
						\item[4.0.5] Nested Lists \dotfill{8}
			\end{enumerate}
			\end{itemize}
			\end{itemize}
			\item[5] \textbf{Pseudocode} \dotfill{8}
			\item[6] \textbf{Coloring} \dotfill{10}
			\item[7] \textbf{Citing the Papers} \dotfill{10}
		\end{enumerate}
		\item[] \textbf{List of Figures}
		\begin{enumerate}
			\item[1] IIT DHARWAD \dotfill{7}
		\end{enumerate}
		\item[] \textbf{List of Tables}
		\begin{enumerate}
			\item[1] A simple table in LaTeX \dotfill{7}
		\end{enumerate}
	\end{itemize}
	\pagebreak
	
	\section{Introduction}
	This document serves as a comprehensive guide to various mathematical and technical concepts, presented in a structured and easy-to-follow format.\\ \\
	\textbf{Section 2: Mathematics} explores fundamental mathematical concepts, beginning with trigonometric equations and identities, and moving through topics
	such as matrices, radicals, integration, summation, differentiation, and the use
	of nested brackets. This section aims to reinforce key principles and provide
	useful equations and formulas for each area. \\ \\
	\textbf{Section 3: Tables and Figures} demonstrates how to include tables and
	figures in a LaTeX document, showcasing ways to organize data and visualize
	information effectively.\\ \\
	\textbf{Section 4: Numbering} explains various types of lists and numbering conventions, including unordered (bullet) lists, ordered (numbered) lists, Roman numerals, alphabetic lists, and nested lists. The section provides examples to
	help readers apply these styles in their own documents. \\ \\
	\textbf{Section 5: Pseudocode} introduces techniques for writing pseudocode in LaTeX, allowing readers to structure algorithms and complex procedures in a clear,readable format. \\ \\
	\textbf{Section 6: Coloring} covers the use of color in LaTeX, showing how to apply
	different colors to text, backgrounds, and other elements to enhance readability
	and aesthetic appeal. \\ \\
	\textbf{Section 7: Citing the Papers}  explains the methods for citing academic
	papers and references in LaTeX, which is essential for creating well-documented
	technical or research papers. \\ \\
	This document includes both a list of figures and a list of tables, helping the
	reader quickly locate visual elements. Each section is carefully structured to ensure clarity and utility, making this guide a valuable resource for anyone seeking
	to enhance their skills in LaTeX and mathematical documentation.
	
	\pagebreak
	
	\section{Mathematics}
	
	\subsection{Writing Trigonometric Equations}
	There are six trigonometric functions:
	\begin{itemize}
		\item \(\sin x\)
		\item \(\cos x\)
		\item \(\tan x\)
		\item \(\cot x\)
		\item \(\sec x\)
		\item \(\csc x\)
	\end{itemize}
	
	\subsubsection{Some trigonometric identities}
	\begin{itemize}
		\item $\sin^2 x + \cos^2 x = 1$
		\item $\sin^2 \theta + \cos^2 \theta = 1$
		\item $1 + \tan^2 x = \sec^2 x$
		\item $1 + \cot^2 x = \csc^2 x$
		\item $\tan(2x) = \frac{2 \tan x}{1 - \tan^2 x}$
	\end{itemize}
	
	\subsection{Matrices}
	\begin{itemize}
		\item Row Vector: A $1 \times N$ matrix:\\
		\(
		\begin{bmatrix}
			a & b & c
		\end{bmatrix}
		\)
		\item Column Vector: A $N \times 1$ matrix:\\
		\(
		\begin{bmatrix}
			a \\
			b \\
			c
		\end{bmatrix}
		\)
		\item Square Matrix: $N \times N$ matrix:\\
		\(
		\begin{bmatrix}
			a & b & c \\
			d & e & f \\
			g & h & i
		\end{bmatrix}
		\)
	\end{itemize}
	
	\pagebreak
	
	\subsection{Radicals}
	\begin{itemize}
		\item Basic Square Roots:\\
		\(\sqrt{a}\)
		\item Square Root with Expression Inside:\\ \(\sqrt{x^2 + y^2}\)
		\item Cube Root:\\
		\(\sqrt[3]{a}\)
		\item nth Root:\\
		\(\sqrt[n]{a}\)
	\end{itemize}
	
	\subsection{Integration}
	The integral of $f(x) = x^2$ from $0$ to $1$:
	\[
	\int_{0}^{1} x^2 \,dx = \frac{1}{3}x^3 \Big|_0^1 = \frac{1}{3}
	\]
	
	\subsection{Summation}
	The sum of the first $n$ natural numbers:
	\[
	\sum_{i=1}^n i = \frac{n(n+1)}{2}
	\]
	\newpage
	\subsection{Differentiation}
	The derivative of $f(x) = x^3$ with respect to $x$:
	\[
	\frac{d}{dx}x^3 = 3x^2
	\]
	
	\subsection{Nested Brackets}
	Example 1:
	\[
	w = \left( \sqrt{(1 + \frac{1}{x^2}}) \cdot \left[ \log \left( 2 + \frac{1}{y} \right) + \left\{ 3^x - \left( \frac{2}{z} \right)^y \right\} \right]\right)
	\]
	Example 2:
	\[
	z = \left\{ \left( \frac{3}{2} + \left[ 5^2 - \left( 4 + \frac{1}{x} \right)^3 \right] \right) \times \left( 1 + \sqrt{y} \right) \right\}
	\]
	
	\pagebreak
	
	\section{Tables and Figures}
	Happy Synonyms
	
	\begin{table}[h]
		\centering
		\begin{tabular}{|c|c|c|} 
			\hline
			Column 1 & Column 2 & Column 3 \\ 
			\hline
			cheerful & delighted & ecstatic \\ 
			\hline
			glad & thrilled & jolly \\ 
			\hline
			jubilant & merry & upbeat \\ 
			\hline
		\end{tabular}
		\caption{A simple table in LaTeX}
	\end{table}
	
	
\begin{figure}
	\centering
	\includegraphics[width=0.5\linewidth]{"iit dharwad"}
	\caption{IIT DHARWAD}
	\label{fig:iit-dharwad}
\end{figure}
	
	\pagebreak
	\section{Numbering}
	
	\begin{enumerate}
		\item[4.0.1]\textbf{Unordered (Bullet) Lists}
		\begin{itemize}
			\item First item
			\item Second item
			\item Third item
		\end{itemize}
		\item[4.0.2]\textbf{Ordered (Numbered) Lists}
		\begin{enumerate}
			\item[1.] First item
			\item[2.] Second item
			\item[3.] Third item
		\end{enumerate}
		
		\item[4.0.3] \textbf{Using Roman Numerals:}
		\begin{enumerate}
			\item[I.] First item
			\item[II.] Second item
			\item[III.] Third item
		\end{enumerate}
		
		\item[4.0.4]\textbf{Using Letters}
		\begin{enumerate}
			\item[a)] First item
			\item[b)] Second item
			\item[c)] Third item
		\end{enumerate}
		\item[4.0.5]\textbf{Nested Lists}
		\begin{itemize}
			\item First item
			\begin{itemize}
				\item Nested item 1
				\item Nested item 2
			\end{itemize}
			\item Second item
		\end{itemize}
	\end{enumerate}
	\section{Pseudocode}
	Pseudocode is an informal way of programming description that does not require any strict programming language syntax or underlying technology considerations. It is used for creating an outline or a rough draft of a program. Pseudocode summarizes a program’s flow, but excludes underlying details. LaTeX has several packages for typesetting algorithms in the form of "pseudocode". They provide stylistic enhancements over a uniform style (i.e., all in typewriter font).
	
	\begin{algorithm}
		\caption{Merge Sort}
		\begin{algorithmic}[1]
			\Procedure{MergeSort}{$A, \text{left}, \text{right}$}
			\If{$\text{left} < \text{right}$}
			\State $\text{mid} \gets \text{floor}((\text{left} + \text{right}) / 2)$
			\State \Call{MergeSort}{$A, \text{left}, \text{mid}$}
			\State \Call{MergeSort}{$A, \text{mid} + 1, \text{right}$}
			\State \Call{Merge}{$A, \text{left}, \text{mid}, \text{right}$}
			\EndIf
			\EndProcedure
			
			\Procedure{Merge}{$A, \text{left}, \text{mid}, \text{right}$}
			\State $n1 \gets \text{mid} - \text{left} + 1$
			\State $n2 \gets \text{right} - \text{mid}$
			\State Create arrays $L[1 \dots n1]$ and $R[1 \dots n2]$
			\For{$i = 1$ to $n1$}
			\State $L[i] \gets A[\text{left} + i - 1]$
			\EndFor
			\For{$j = 1$ to $n2$}
			\State $R[j] \gets A[\text{mid} + j]$
			\EndFor
			\State $i, j, k \gets 1, 1, \text{left}$
			\While{$i \leq n1$ and $j \leq n2$}
			\If{$L[i] \leq R[j]$}
			\State $A[k] \gets L[i]$
			\State $i \gets i + 1$
			\Else
			\State $A[k] \gets R[j]$
			\State $j \gets j + 1$
			\EndIf
			\State $k \gets k + 1$
			\EndWhile
			\While{$i \leq n1$}
			\State $A[k] \gets L[i]$
			\State $i \gets i + 1$
			\State $k \gets k + 1$
			\EndWhile
			\While{$j \leq n2$}
			\State $A[k] \gets R[j]$
			\State $j \gets j + 1$
			\State $k \gets k + 1$
			\EndWhile
			\EndProcedure
		\end{algorithmic}
	\end{algorithm}
	
	\newpage 
	\section{Coloring}
	
	{\color{blue} This is blue text.}\\
	{\color{red} This is red text.}\\
	\noindent
	\hl{This text has a yellow background.}\\
	{\color{purple} This entire paragraph is purple. All sentences here will appear in purple.}
	
	\section{Citing the Papers}
	\begin{itemize}
		\item Investigating the impact of network topology on the processing times of SDN controllers \cite{metter2015investigating}
		\item SDN controllers: A comparative study \cite{salman2016sdn}
		\item Controllers in SDN: A Review Report \cite{paliwal2018controllers}
		\item Software defined networks: Comparative analysis of topologies with ONOS \cite{rajaratnam2017software}
	\end{itemize}
	
	\begin{thebibliography}{9}
		\bibitem{metter2015investigating}
		C. Metter, S. Gebert, S. Lange, T. Zinner, P. Tran-Gia, and M. Jarschel,
		``Investigating the impact of network topology on the processing times of sdn controllers,'' in \textit{2015 IFIP/IEEE International Symposium on Integrated Network Management (IM)}, pp. 1214--1219, 2015.
		
		\bibitem{salman2016sdn}
		O. Salman, I. H. Elhajj, A. Kayssi, and A. Chehab, ``Sdn controllers: A comparative study,'' in \textit{2016 18th Mediterranean Electrotechnical Conference (MELECON)}, pp. 1--6, 2016.
		
		\bibitem{paliwal2018controllers}
		M. Paliwal, D. Shrimankar, and O. Tembhurne, ``Controllers in sdn: A review report,'' \textit{IEEE Access}, vol. 6, pp. 36256--36270, 2018.
		
		\bibitem{rajaratnam2017software}
		A. Rajaratnam, R. Kadikar, S. Prince, and M. Valarmathi, ``Software defined networks: Comparative analysis of topologies with onos,'' in \textit{2017 International Conference on Wireless Communications, Signal Processing and Networking (WiSPNET)}, pp. 1377--1381, 2017.
	\end{thebibliography}
	
\end{document}
